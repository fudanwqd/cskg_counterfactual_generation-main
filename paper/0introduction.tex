
\section*{Introduction}
This template was made for the course introduction to data science \cite{IDS2017}. 
In this section we explain some of the conventions which should be followed for answering theory questions, and how you should organize the report when explaining the experimental part of the 
assignment. Here are some general rules of the report who have to hand in:
\begin{itemize}
 \item You are expected to \textbf{read and follow} instructions via Nestor and Github. Adhere to the rules and conventions of this report template.
 \item The answers to the assignment questions should have number codes as the questions being answered. Students are requested not to change the number codes. For example, if one is writing the 
answer of question 2.1 sub part (a), the answer should have the tag 2.1(a). If the answer to the question 2.1(a) is tagged as II.(i)(a) or 2.1.A or (ii).I.(a) or anything which is different 
from the convention followed in numbering the assignment questions, the answers might be missed and not graded.
\item Do not exceed the page number limit of 5pages , single column.
\item Do not change the font size and page font.
\item Do not change the page margins.
\item Provide snippets of codes in the main report only when \textbf{explicitly} asked for it. Else provide code in Appendices only if you really want to share. 
\item We ask for the codes to be submitted separately anyway. We will run these codes and only if they produce the same figures as those in your report, will your report be validated.
\end{itemize}
The following part guides you on how you should organize the report when answering the practical parts of the assignment, where you have to perform experiments.
\begin{enumerate}
 \item Start with the correct number and letter tag of the question of the assignment.
 \item Provide a \textbf{Motivation} for the experiments [\textit{Had to do it for IDS assignment} or \textit{because the question asked me/us to do} will not be accepted as motivation]. 
 \item Mention the problem statement, where you point out exactly which sub-sets of the issues associated with the \textit{Motivation} you are trying to investigate, and why.
 \item Give a brief overview of what are the available resources and tools which have have in the past and recent past been applied to find solutions to the problems you are trying to handle. 
 \item Explain how your experiment report is organized, i.e., you mention that:
 \begin{itemize}
  \item the first part is the \textbf{Introduction} to the topic, which contains the motivation, problem statement and literature review briefly.
  \item the second part is the \textbf{Methodology} where you have explained briefly the \textit{n} techniques you have used in the concerned experiment.
  \item the third part is the \textbf{Results} where you have put your figures, tables, graphs, etc. 
  \item the fourth part is \textbf{Discussion} where you have discussed the results.
  \item the report ends with a \textbf{Conlcusion} where you have concluded about your findings 
 \end{itemize}

\end{enumerate}